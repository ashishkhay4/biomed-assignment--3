\documentclass[12pt]{article}

\usepackage{graphicx}
\graphicspath{{ASHISH}}

\begin{document}

\begin{figure}
\centering
\includegraphics[scale=1]{collage.png}

\end{figure}

\section*{ASSIGNMENT -3 of BioMedical Engineering\\"FUTURE OF HEALTH CARE"\\\\Submitted by : ASHISH\\Roll.no:21111013\\Details: 1st Semester, Biomedical Engineering\\\\Supervision by : SAURABH GUPTA SIR }

\clearpage
\section*{INTRODUCTION :}
The future of health will likely be driven by digital transformation enabled by radically interoperable data and open, secure platforms. Health is likely to revolve around sustaining well-being rather than responding to illness. ... We expect prevention and early diagnoses will be central to the future of health.

\section*{3 Ways Technology is Changing Health care :}
\section*{1 .NanoTechnology :}
We are living at the dawn of the nanomedicine age. I believe that nanoparticles and nanodevices will soon operate as precise drug delivery systems, cancer treatment tools or tiny surgeons.

As the technology evolves, we will see more practical examples of nanotechnology in medicine. Future PillCams could even take biopsy samples for further analysis while remote-controlled capsules could make the prospect of nano-surgeons a reality.
\centering
\includegraphics[scale=0.2]{nano.jpg}
\section*{2 .ROBOTICS :}
robotics, design, construction, and use of machines (robots) to perform tasks done traditionally by human beings.  
robotics, design, construction, and use of machines (robots) to perform tasks done traditionally by human beings.  Robots are widely used in such industries as automobile manufacture to perform simple repetitive tasks, and in industries where work must be performed in environments hazardous to humans.

\section*{3. 3D-Printing :}
3D-printing can bring wonders in all aspects of healthcare. We can now print biotissues, artificial limbs, pills, blood vessels and the list goes on and will likely keep on doing so.The pharmaceutical industry is also benefiting from this technology. FDA-approved 3D-printed drugs have been a reality since 2015 and researchers are now working on 3D-printing “polypills”.
\centering
\includegraphics[scale=0.2]{3d-image.jpg}

\section*{4.  CONCLUSION :}
Digital revolution undoubtedly modifies the way we develop, practice, and provide medicine. This paradigm shift will directly influence the evolution of health-care systems. Technology allows a more and more precise and personalized medicine. However, overcontrol of health could lead to a new scary biopolitical power. Patients should stand at the heart of the Healthcare System. Technologies that long term benefit to the patient will be accepted. Human relationship and empathy still remain essential.

\end{document}